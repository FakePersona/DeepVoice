\documentclass[11pt,english]{beamer}

\usepackage[T1]{fontenc}
\usepackage[utf8]{inputenc}
\usepackage[main=french,english]{babel}
\usepackage{hyperref}
\usepackage{tikz}
\usetikzlibrary{shapes,snakes}
\usepackage{ulem}
\usepackage{xcolor, colortbl}
\usepackage{mathtools}

\usetheme{AnnArbor}
\usecolortheme{beaver}

\title{DeepVoice}
\subtitle{Extracting meaningful signal representation for Speaker Recognition
  using deep architectures}

\author{Rémi~Hutin, Rémy Sun, Raphaël Truffet\\ \\Encadrants : Guillaume Gravier et Vedran Vukoki\'c}


\institute{ENS Rennes, IRISA}

\date{8 décembre 2016}


\AtBeginSection[]{
    \begin{frame}
        \frametitle{Sommaire}
        \tableofcontents[currentsection,hideothersubsections]
    \end{frame} 
}


\begin{document}

\begin{frame}
    \titlepage
\end{frame}

\begin{frame}
    \frametitle{Sommaire}
    \tableofcontents[hideallsubsections]
\end{frame}



\section{Figures}

\begin{frame}
\def\layersep{2.5cm}


\begin{figure}
\centering
\begin{tikzpicture}[shorten >=1pt,->,draw=black!50, node distance=\layersep]
    \tikzstyle{every pin edge}=[<-,shorten <=1pt]
    \tikzstyle{neuron}=[ellipse,fill=black!25,minimum size=17pt,inner sep=0pt]
    \tikzstyle{input neuron}=[neuron, fill=green!50];
    \tikzstyle{output neuron}=[neuron, fill=red!50];
    \tikzstyle{hidden neuron}=[neuron, fill=blue!50];
    \tikzstyle{annot} = [text width=4em, text centered]

    % Draw the input layer nodes
    \foreach \name / \y in {1,...,4}
    % This is the same as writing \foreach \name / \y in {1/1,2/2,3/3,4/4}
        \node[input neuron, pin=left:] (I-\name) at (0,-\y) {$a_\y$};

    % Draw the hidden layer nodes
    \foreach \name / \y in {1,...,1}
        \path[yshift=-1.5cm]
            node[hidden neuron] (H-\name) at (\layersep,-\y cm) {};

    % Draw the output layer node
    \node[output neuron, right of=H-1] (O) {$e = f(WA + b)$};

    % Connect every node in the input layer with every node in the
    % hidden layer.
    \foreach \source in {1,...,4}
        \foreach \dest in {1,...,1}
            \path (I-\source) edge (H-\dest);

    % Connect every node in the hidden layer with the output layer
    \foreach \source in {1,...,1}
        \path (H-\source) edge (O);

    % Annotate the layers
    \node[annot,below of=H-1, node distance=1cm] (hl) {Weight $W$\\Bias $b$};
    \node[annot,above of=H-1, node distance=1cm] (hl) {Neuron};
    \node[annot,above of=I-1, node distance=1cm] (hl) {Input A};
\end{tikzpicture}

\end{figure}

\end{frame}






\begin{frame}
\def\layersep{1.8cm}


\begin{figure}
\centering
\begin{tikzpicture}[shorten >=1pt,->,draw=black!50, node distance=\layersep]
    \tikzstyle{every pin edge}=[<-,shorten <=1pt]
    \tikzstyle{neuron}=[circle,fill=black!25,minimum size=17pt,inner sep=0pt]
    \tikzstyle{input neuron}=[neuron, fill=green!50];
    \tikzstyle{output neuron}=[neuron, fill=red!50];
    \tikzstyle{hidden neuron}=[neuron, fill=blue!50];
    \tikzstyle{hidden neuron back}=[neuron, fill=orange!50];
    \tikzstyle{annot} = [text width=4em, text centered]

    % Draw the input layer nodes
    \foreach \name / \y in {1,...,4}
    % This is the same as writing \foreach \name / \y in {1/1,2/2,3/3,4/4}
        \node[input neuron, pin=left:] (I-\name) at (0,-\y) {$a_\y$};

    % Draw the hidden layer nodes
    \foreach \name / \y in {1,...,5}
        {
        \visible<-10>{\path[yshift=0.5cm]
            node[hidden neuron] (H1-\name) at (\layersep,-\y cm) {};}
        \visible<11->{\path[yshift=0.5cm]
            node[hidden neuron back] (H1-\name) at (\layersep,-\y cm) {};}
        }
            
    \foreach \name / \y in {1,...,4}
        {
        \visible<-9>{\path
            node[hidden neuron] (H2-\name) at (2*\layersep,-\y cm) {};}
        \visible<10->{\path
            node[hidden neuron back] (H2-\name) at (2*\layersep,-\y cm) {};}
        }
            
    \foreach \name / \y in {1,...,5}
        {
        \visible<-8>{\path[yshift=0.5cm]
            node[hidden neuron] (H3-\name) at (3*\layersep,-\y cm) {};}
        \visible<9->{\path[yshift=0.5cm]
            node[hidden neuron back] (H3-\name) at (3*\layersep,-\y cm) {};}
        }
    % Draw the output layer node
    \node[output neuron,pin={[pin edge={->}]right:Output}, right of=H3-3] (O) {};

    % Connect every node in the input layer with every node in the
    % hidden layer.
    
    \foreach \source in {1,...,4}
        \foreach \dest in {1,...,5}
            {\only<\source>{\path (I-\source) edge (H1-\dest);}}
            
            
    \visible<5->{
    \foreach \source in {1,...,4}
        \foreach \dest in {1,...,5}
            \path (I-\source) edge (H1-\dest);
    }
    
            
    \visible<6->{
    \foreach \source in {1,...,5}
        \foreach \dest in {1,...,4}
            \path (H1-\source) edge (H2-\dest);
    }
            
    \visible<7->{
    \foreach \source in {1,...,4}
        \foreach \dest in {1,...,5}
            \path (H2-\source) edge (H3-\dest);
    }
            
    \visible<8->{
    \foreach \source in {1,...,5}
        \path (H3-\source) edge (O);
    }
        
        
    % Annotate the layers
    \node[annot,above of=H1-1, node distance=1cm] (hl1) {Hidden layer 1};
    \node[annot,right of=hl1] (hl2) {Hidden layer 2};
    \node[annot,right of=hl2] (hl3) {Hidden layer 3};
    \node[annot,right of=hl3] {Output layer};
    \node[annot,left of=hl1] {Input layer};
    
    \node[annot,below of=H1-5, node distance=1cm] (wb1) {$W\visible<11->{'}_1$, $b\visible<11->{'}_1$};
    \node[annot,right of=wb1] (wb2) {$W\visible<10->{'}_2$, $b\visible<10->{'}_2$};
    \node[annot,right of=wb2] (wb3) {$W\visible<9->{'}_3$, $b\visible<9->{'}_3$};
\end{tikzpicture}

\end{figure}

\end{frame}

\section{Signal representation for speaker recognition}

\begin{frame}
  \frametitle{Cesptral alanysis}
  
\end{frame}

\begin{frame}
  \frametitle{Cepstral space}
  
\end{frame}

\begin{frame}
  \frametitle{Probabilistic modeling}
  
\end{frame}

\begin{frame}
  \frametitle{supervectors}
  
\end{frame}

\begin{frame}
  \frametitle{i-vectors}
  
\end{frame}

\begin{frame}
  \frametitle{i-vectors}
  
\end{frame}

\begin{frame}
  \frametitle{i-vectors}
  
\end{frame}

\begin{frame}
  \frametitle{transition}
  
\end{frame}

\begin{frame}
  \frametitle{transition}
  
\end{frame}

\section{Deep learning}

\begin{frame}
  \frametitle{Use of DNNs}
  
\end{frame}

\begin{frame}
  \frametitle{Formal neuron}
  
\end{frame}

\begin{frame}
  \frametitle{Neural network}
  
\end{frame}

\begin{frame}
  \frametitle{Autoencoder}
  
\end{frame}

\begin{frame}
  \frametitle{Autoencoder}
  
\end{frame}

\begin{frame}
  \frametitle{i-vector 2.0}
  
\end{frame}

\section{Methodology}

\begin{frame}
  \frametitle{Supplanting i-vectors}
  
  \begin{columns}
  \column{0.44\textwidth}  
  
\only<1>{
\begin{equation*}
    \begin{bmatrix}
      G_0^{0,0} \\   G_0^{0,1} \\ ... \\ G_{0}^{0,255} \\ G_0^{1,0} \\ .. \\ G_{0}^{N,255}
    \end{bmatrix}
\begin{bmatrix}
      G_1^{0,0} \\   G_1^{0,1} \\ ... \\ G_{1}^{0,255} \\ G_1^{1,0} \\ .. \\ G_{1}^{N,255}
    \end{bmatrix}
...
\begin{bmatrix}
      G_M^{0,0} \\   G_M^{0,1} \\ ... \\ G_{M}^{0,255} \\ G_M^{1,0} \\ .. \\ G_{M}^{N,255}
    \end{bmatrix}
  \end{equation*}
M supervectors
}

\only<2>{
  \begin{equation*}
    \begin{bmatrix}
      G_0^{0,0} \\   G_0^{0,1} \\ ... \\ G_{0}^{0,255} \\ G_0^{1,0} \\ .. \\ G_{0}^{N,255}
    \end{bmatrix}
  \end{equation*}
  Supervector from signal spoken by \textbf{someone}
}
  \column{0.5\textwidth}  
  What is needed to extract i-vectors ?
  \begin{itemize}
  \item Supervectors
  \item Training data (Labels on supervectors)
  \end{itemize}
  We will use the \textbf{exact same data}
  \end{columns}

\end{frame}

\begin{frame}
  \frametitle{Processed data}

  \begin{columns}
    \column{0.6\textwidth}
    \begin{itemize}
\setlength\itemsep{2em}
    \item \textbf{Raw data:} 15311 numeric soud files from BFMTV with labeled speakers
    \item \textbf{Pre-processed data:} 3 678 470 pairs $(v_1,v_2)$ of supervectors
      spoken by the same person
    \item \textbf{Input:} Supervector $v_1$ of length 9216
    \item \textbf{Output:} Supervector $v_2$ of length 9216

    \end{itemize}

    \column{0.35\textwidth}
  \end{columns}
  
\end{frame}

\begin{frame}
  \frametitle{Intermediate vector evaluation}
  
\end{frame}

\begin{frame}
  \frametitle{Intermediate vector evaluation}
  
\end{frame}

\section{Discussion}

\begin{frame}
  \frametitle{Goal}
  
\end{frame}

\begin{frame}
  \frametitle{Expected issues}
  
\end{frame}

\end{document}



